%%
%% $Id$
%% 

\chapter{Description of the MScheme System}

{\bf NOTE: The source for this document is available in the directory:}\par
\hskip4pc{\tt mscheme/doc/}

\bigskip

\section{Overview}

This document describes the design and implementation of the
Modula-Scheme development environment, or MScheme for short.  The
MScheme system consists of several components and is intended for
several uses.  In the simplest view, the MScheme system is simply an
interpreter for the Scheme\cite{scheme-initial-report,r4rs}
programming language, which happens to be coded in the
Modula-3\cite{spwm3} language.  This is not a completely satisfying
description, however, as it does not capture those aspects of the
system intended to be novel.

The MScheme system intends to innovate in two separate but intertwined
areas: embedded interpreters and distributed agent systems.  In short,
the MScheme system aims at bringing the same kind of inter-language
bindings possible in a two-level interpreted environment\footnote{A
  two-level interpreted environment is one in which an interpreter has
  been coded, or {\em embedded\/}, in another interpreted language in
  such a way that the embedded language has access to all the
  facilities of the implementation language } to an environment based
on a traditionally compiled systems language; it aims to do this in a
way that, when necessary, allows the system integrator to have so much
control over the interpreted environment that untrusted third-party
code can be allowed to run in the interpreter.

\section{Intended Uses of the MScheme System}

The MScheme system is intended to be usable in several very different
ways.  Some users may choose to use it simply as a convenient Scheme
interpreter.  It is also intended to be easy to use as a ``glue''
interpreter for building large Modula-3 applications out of components
and to replace special-purpose interpreters built into many large
Modula-3 applications.  Finally, and perhaps most interestingly,
MScheme interpeters are intended to be usable as ``agent
environments'' for code run that can be uploaded into an application
by a remote user.  Depending on the application, such code can either
be trusted and hence given full access to Modula-3 facilities, or else
untrusted and hence given only restricted access to Modula-3
facilities.

\section{Components of the MScheme System}

The MScheme system consists of a number of software components written
both in Modula-3 and Scheme.  In order for the project to achieve all its
goals, it will also include some components not yet written.

\subsection{The MScheme interpreter}

The basic component of the MScheme system is a Scheme interpreter
written in Modula-3.  This Scheme interpreter is closely based on
Peter Norvig's final version of the JScheme
interpreter\cite{jscheme}.\footnote{The current JScheme interpreter
  available from {\tt sourceforge.net} is a highly developed system
  that has achieved much higher performance than Norvig's original
  version but without really fixing some of the basic deficiencies of
  JScheme.  We use Norvig's old version as our starting point simply
  because there is less code in it to fix.}  JScheme is an interpreter
that Norvig wrote in around 1997 to teach himself Java.  JScheme is
well embedded in Java: it allows the {\tt new}ing of any Java object
and it also allows the calling of any named method on any such Java
object.

MScheme started out as a direct translation of JScheme into Modula-3.
The fact that Modula-3 does not permit as extensive
introspection\footnote{{\em Introspection\/} is the process whereby a
  program can analyze its own code; it is a facility available to a
  greater or lesser extent in certain dynamically typed languages like
  Smalltalk\cite{blue_book} but generally not in languages with
  traditional compiler architectures like C, Fortran, or C++.
  Modula-3 permits limited introspection (mainly of certain aspects of
  the type system), but not enough to be useful to us in the MScheme
  system.} as Java means that in order to provide access to all
Modula-3 facilities from MScheme, a special {\em stub generator\/}
program has to be provided, which will be described in section
\ref{sec:stub_generator} below.

\subsubsection{Deficiencies of the JScheme interpreter}

A few aspects of Norvig's JScheme system reflect his desire to write
what is close to the smallest Scheme interpreter possible in Java at
the expense of some usability.  Norvig's Scheme interpreter was
written in the ``Web 1.0'' days, while Java was still thought to be
the language by which Web browsers would gain more functionality
through the so-called {\em web applets\/}\footnote{Web applets are
  untrusted code intended to be downloaded by a web browser from a web
  server and run in a controlled environment so that they cannot cause
  any damage on the user's computer.  This was intended to be the way
  that powerful applications would be provided via the web, but has
  been superseded by inferior technoligies such as JavaScript.}
Because of Norvig's intent to embed his Scheme interpreter in Web
pages, he was very eager to make the interpreter as small as possible,
and its original name, SILK, is a pun on this, ``Scheme in L K'',
where L is the Roman numeral for decimal 50.  As a result, JScheme has some
serious limitations, some of which seem to be (somewhat puzzlingly) present
in the latest versions of JScheme as well, which have long since abandoned
the ``smallest Scheme interpreter possible in Java'' goal.

Some limitations of JScheme are the following.
\begin{itemize}
\item JScheme offers an all-or-nothing (really an ``all'') approach to
  interfacing with Java.  Scheme code is allowed to make any and all
  Java calls.  This is perhaps acceptable in a web-applet environment,
  in which the Java interpreter enforces strict access control to
  system resources, but it is not at all sufficient for use in the
  ``reverse'' environment, where untrusted Scheme code is uploaded
  into a server-side {\em application.}  It is easy enough to remove
  the Scheme interpreter's ability to make {\em any\/} Java calls but
  that is throwing the baby out with the bath water.

\item JScheme offers no specific mechanism for extending the Scheme
  interpreter.  Again,this may be acceptable as long as one provides a
  simple mechanism for accessing any Java facilities whatsoever, but
  it is not acceptable if one's goal is to provide access to a limited
  list of external facilities.

\item JScheme has no specific design details that address
  multi-threading.  Specifically, JScheme makes no provision for two
  threads' concurrently accessing a single Scheme environment nor of
  having two interpreter that share a single Scheme environment.

\item JScheme is exceedingly inefficient with its memory allocations,
  which puts a heavy load on the Java garbage collector.

\item JScheme maintains Scheme environments as a Java data structure,
  but it uses the Java stack for {\em control\/}.  This means that the
  only control facilities that can be provided in the interpreter are
  a subset of Java's control facilities.  In particular, full Scheme
  continuations cannot be implemented in Java.  (Unwinding Scheme
  continuations are provided in JScheme by means of an exception.)
  This choice of implementation also implies high memory use for stacks
  in a multi-threaded environment, since the stacks have to be sized
  large enough to accomodate all the control levels of the Scheme program.
  When a Scheme program is not properly tail recursive this is especially
  problematic as even modest programs can run out of stack space.

\end{itemize}

Of the issues raised in the above list, MScheme has addressed all but
the last as of this writing.  MScheme has a library interface for
extending the interpreter (see
section~\ref{sec:extending_the_interpreter}), and it has a new type of
synchronized, ``navigable'' environment system that allows several
interpreters to cooperate via shared environments (see
section~\ref{sec:navigable_environments}). The MScheme implementation
also recycles memory aggressively and uses small arrays instead of
linked lists to represent very small environments, both of which
greatly reduce the load on the garbage collector.  On the same type of
processor, MScheme currently runs several times faster than the version
of JScheme upon which it was based.  The only point on the list that 
remains to be addressed is the ``stacked'' implementation of the interpreter
(see section~\ref{sec:future_enhancements});

\subsection{The Modula-3 analyzer and stub generator}\label{sec:stub_generator}

\subsection{Interactive environments}

Two interactive, ``bare'', interpreters are provided, {\tt
  mscheme/interactive}, a basic Scheme interpreter with a terminal
interface and {\tt mscheme/interactive\_r}, the same interpreter with a
GNU Readline interface.

%%%%%%%%%%%%%%%%%%%%

\section{Future Enhancements}\label{sec:future_enhancements}

\subsection{Concurrency}

The present implementation of the system provides thread-safe
environments as an option.  However, all this accomplishes is that two
Scheme threads can safely share variables.  Synchronization in the
Scheme code itself would have to be provided via Dekker's or
Peterson's algorithms, which is not very convenient.  Allowing direct
access from Scheme to locks or mutexes is not really an acceptable
solution, as this would permit Scheme programs to deadlock the
Modula-3 system (and cause a system crash).  In keeping with the areas
where we expect to see the system used, it would be acceptable to
allow Scheme programs to deadlock, but it should not cause resource
starvation for other programs if that happens (some sort of abortion
and garbage collection should be provided).  Also the correct
synchronization primitives must be chosen out of the zoo of
possibilities---something like a functional monitor might be elegant
and Scheme-like.\footnote{The original Scheme system was intended to
  be run on a concurrent system, a hardware implementation of Scheme,
  in fact, but the synchronization primitives provided in that system
  are somewhat inconvenient.}

\subsection{Enhancements required in the interpreter}

The use of the Modula-3 stack to push Scheme control has to be
eliminated.  One straightforward way of doing this is to code the
interpreter in a {\em continuation-passing style (CPS)\/}.  The Scheme
stack would in such a system be implemented directly as a Modula-3
data structure.  Many benefits would accrue from such a design change.
First, Modula-3 stack would not be pushed when the Scheme stack is,
which would allow much higher recursion depths in Scheme, limited only
by the available heap space, which is much better managed than the
stack space by the Modula-3 runtime (this is true for most if not all
contemporary systems programming languages).  Secondly, the move from
stack to heap memory would make it much easier to set specific limits
on how much ``stack'' a particular instance of Scheme is allowed to
push (mostly because there is enough available to set a reasonable
limit, rather than having to limit it to something unreasonably
small!)  Lastly, the described change would make it possible to implement
Scheme's {\tt call-with-current-continuation}, which is, after all,
one of the main innovations of the Scheme language.


\section{Running Untrusted Code}

One important way of using MScheme is permitting an untrusted user to
provide Scheme code that can be run inside an application.  Such code
might, for instance, be uploaded over the Internet and used in
e-commerce or some other application that requires distributed
decision making and reliable code.  Some of the author's earlier
experiences (see section~\ref{sec:prior_work}) have indicated that
being able to run third-party untrusted code in a powerful environment
such as Scheme has a number of potential pitfalls.  The following is
a short summary of the steps that must be taken.
\begin{itemize}

\item Obviously, the system must not, indirectly or directly, provide
  access to untrusted low-level primitives (e.g., write a file at an
  arbitrary path on the server system).  MScheme can be put in a mode
  where it does not provide access to such primitives simply by not
  loading untrusted primitives (the default interpreter does not provide
  any).

\item The system must not permit a single untrusted third-party
  application to crash more than the parts of the system belonging to
  the user uploading the application.  This can be easier said than
  done!  Obviously it means that certain categories of bugs must be
  ruled out of the system, but less obviously, the system must not
  allow a single untrusted program to use up a large amount of
  resources (especially memory).  Unfortunately this objective is
  difficult to attain as long as the Scheme interpreter runs on the
  Modula/Java stack, which sets a fairly low limit on the amount of
  stack that can be pushed before a crash.  It is on the other hand
  fairly easy to ensure that a buggy or looping Scheme program does not
  cause any adverse effects (the author has coded a version of JScheme
  that has these properties, see section~\ref{sec:prior_work}.

\item Continuing in the vein of the previous point, there can be no
  non-terminating loops in the interpreter itself.  Again, this is a
  bit more difficult to ensure than is obvious at first sight.  The
  main problem is that Scheme interpreters are notorious for getting
  into infinite loops when performing certain actions (e.g., printing)
  on circular data structures.  If a loop (in the interpreter!) cannot
  be proved to terminate, it {\em must\/} have a facility for
  detecting that it is looping at runtime, e.g., breaking off if a
  timer has expired or if it can otherwise be prove that the code is
  looping.

\item The most difficult issues can occur in the application itself.
  Sometimes it is difficult to know what aspects of the application
  can be revealed to third parties and which might constitute
  ``information leakage'', especially in a financial-services
  environment.

\end{itemize}


\section{Background and Prior Work}\label{sec:prior_work}

\subsection{History of the MScheme system}

The MScheme system was originally conceived as a Modula-Smalltalk
combination, to be used principally for enhancing Modula-3
applications used for the financial software activities of Generation
Capital, Ltd.  The author of the present paper produced a white paper
on the design of the Modula-Smalltalk system and proceeded to
implement some of the basic components of the system.  This effort was
redirected towards Python, mainly because Python is more widely used
today than Smalltalk, and also Smalltalk has a few design decisions
that reflect its age, mostly in the area of concurrency, which is via
cooperative multitasking.  The intention with Modula-Python was to
start with the existing C implementation of Python and link it
together with Modula-3, using the fact that Modula-3 and C have the
same function call conventions (on most systems).  However, the
difficulty in combining the languages in this way is to make the
Modula-3 and Python runtime systems respect each other: the difficulty
here is mainly in the area of garbage collection.  (It must be said
that an enterprising programmer could definitely port Python-in-Java,
Jython, to Modula-3 just as we have with JScheme.)  

At the same time that work at Generation Capital was revolving around
discussions on how to combine Modula-3 and Smalltalk or Python, the
author was involved in a marketing project at a startup company
in Southern California.  This project involved the use of JScheme to
provide a safe environment for third-party agents to run in and to
access an underlying database system.  The experiences with this system
were so positive that the author decided to port JScheme to Modula-3.

The decision to use Scheme rather than Smalltalk or Python has one
drawback, namely that while Smalltalk and Python are both
object-oriented languages with type systems similar to Modula-3's,
Scheme is not explicitly an object-oriented language.  It is however
probably true that by the use of Scheme's powerful macro system and
its first-class procedures, a convienently usable object-oriented
system can be provided to MScheme programmers.  Several existing
systems (Tiny-CLOS, ooo) provided object orientation in Scheme
already.  The advantage of using Scheme is, of course, that Scheme is
a much smaller language than either Python or Smalltalk and it is
therefore far easier both to provide the initial interpreter and to
verify that it meets the necessary properties for the intended
applications.

The MScheme system is currently being used to implement one 
substantial software system, an order-management system for stock market
trading.

\subsection{Prior work}

The idea of embedded languages is hardly a new one.  Embedded
languages has been the way that Lisp programmers have come up with new
versions of Lisp ever since that language was invented some fifty
years ago.  However, the use of embedded languages in ``traditional''
systems programming environments such as C has a shorter history.
Recently, the GNU project has chosen Scheme to be its standard
``scripting'' language, via the GNU Guile interpreter.  In specialized
applications, the same sort of argument has been used previously:
Cadence, Inc. has provided a scripting language called SKILL (also
based on Scheme) in its VLSI design tools for many years.  Scheme is
also the scripting language of the Magic VLSI layout editor.  A
language with a slightly longer history is the language Tcl,
originally invented in the early 1990s by Ousterhout et al. at
Berkeley, also with the goal of scripting VLSI design tools.

The idea that it is possible to put together a scripting language and
a systems language ``transparently'' has an even shorter history than
the VLSI design tools experience, and started with several projects
like JScheme in the late 1990s.  Much later, Sun started providing
support for such scripting language access to Java.  Microsoft is as
of this writing still working on exactly how to do it within their
.NET framework.

The approach we have taken is different from the prior approaches in two
different ways: we specifically have chosen a {\em simple\/} language,
Scheme, in order that we can hope to verify very high-level properties
of the resulting system, which would be necessary for any real application
involving untrusted third-party code; also, the MScheme system is the
first time to our knowledge that the transparent embedding of a scripting
language has been done in a systems programming language that relies on
a traditional compiler architecture.  Java, for instance, provides its
scripting interface via its advanced introspection facility.  By processing
the Modula-3 source itself, the MScheme system obviates the need for such
introspection.

It is important to note that many of the features that make it
possible to implement MScheme and to provide a transparent interface
to Modula-3 facilities depend on having an underlying environment that
has features that systems languages have only been slowly accepting
of, e.g., garbage collected storage and runtime access to type information.  It
would be very difficult, for instance, to implement MScheme in a C++
environment.
