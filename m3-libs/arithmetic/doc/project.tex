% extracted from arch.tex
% currently not of interest due to small user and developer base
\section{Project Architecture}
\subsection{Team Structure}
\begin{description}
\item[Project Lead]
     Coordinates all aspects of the project.
     
\item[Project Architect]
     Approves the initial definition of and the subsequent
     modifications of the public interface, the directory
     structure, and the sufficiency of the supporting
     analysis.
     
\item[Domain Expert]
     Identifies topic area, researches it, develops
     candidate interfaces, writes the analysis, codes the
     test module and the implementation module.
     Periodically, the domain expert reviews his/her work
     with other team members and makes appropriate
     modifications.
     
\item[Publicist]
     Maintains web page, coordinates newsgroup postings,
     coordinates public announcements and discussions of the
     project.
\end{description}

\subsection{Communications}
The team communicates via email and via postings to the
project's web page under
\begin{verbatim}
    http://www.eskimo.com/~hgeorge/m3na
\end{verbatim}

Text must be readable in raw ascii form (e.g., as a simple
insertion into an email message.) The formal documentation is in \TeX.

Generally, a domain expert's analysis needs to be understood
and approved by at least one other team member before the
implementation code is reviewed.  The code, and its test
modules, must be reviewed by at least one other team member
prior to release.

Once the project is established, releases will be made via
M. Dagenais's ftp site.

\subsection{Work Breakdown Structure (WBS)}
\begin{verbatim}
     1.                Manage project
          1.1.         Develop general documentation
          1.2.         Maintain web site
          1.3.         Coordinate analysis and design reviews
          1.4.         Coordinate code reviews
     2.                Develop na.i3 and nap.i3 skeletons
     3.                Develop release <rr>
          3.1.         Identify domain experts and select topics
          3.2.         Develop topic <tt>
               3.2.1.  Research the field, and discuss opportunities
               3.2.2.  Develop test module
               3.2.3.  Perform original analysis, prototype as needed
               3.2.4.  Discuss and approve anaysis.
               3.2.5.  Code up alternatives
               3.2.6.  Test for accuracy
               3.2.7.  Test for timing
               3.2.8.  Review selected alternative with team
               3.2.9.  Stabilize interface, and retest
          3.3.         Publish for external review
          3.4.         Package and release
          3.5.         Perform regression tests
          3.6.         Update user's guide
          3.7.         Update developer's guide
          3.8.         Perform beta tests (if testers are available)
          3.9.         Publicize
\end{verbatim}

