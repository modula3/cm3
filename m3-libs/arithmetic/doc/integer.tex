\section{xInteger: Integers}

\subsection*{Types}
The various short cardinals and integers are for convenience.  The
Array is used as an index array several places in the library.

\subsection*{fmt, fmtArray}
Convenience functions.

\subsection*{Factoring: isprime}
This is defined in the xPrime module, but accessed from xInteger.  [HGG: At one
time it was implemented in the xInteger module, but compiling the
bitmap took too long.]

xPrime is mainly a bitmap of the cardinals from
0..46933.  Primes are marked 1 and nonprimes are marked 0.  The array is
0-based for processing convenience.  46933 was chosen as the sqrt of
the largest CARDINAL ($46933 > \sqrt{2^{31}}$).  In factoring problems,
we will not get anything larger than this.

The primes themselves were obtained from Project Gutenberg's prime11.txt
list.  They were converted to the bitmap formatrix via a perl script:

\begin{tt} \begin{verbatim}
#file:     mkprime.pl
#abstract: Make a bitstring of primes
#          for use in Modula-3
#
($sec,$min,$hour,$mday,$mon,$year,$wday,$yday,$isdst) = localtime(time);
print "$0 $mon/$mday/$year  $hour:$min\n";
print "Copyright (c) 1996, Harry George\n";

#---configuration---
$inname ="prime11.txt";
$outname="prime.m3";
$linewidth=60;
$maxline=4850;

#---open files---
open(INFILE,"<$inname") || die "cannot open $inname\n";
open(OUTFILE,">$outname") || die "cannot open $outname\n";


#---skip to line 300---
for ($i=0; $i<299; $i++) {
  $line=<INFILE>;
};

#---read data into array---
for ($i=1; $i<$maxline; $i++) {
  $line=<INFILE>; chop($line);
  push(@primes,$line);
};    
$minprime=$primes[0];
$maxprime=$primes[$#primes];

#---generate the bitarray---
###select(STDOUT);
select(OUTFILE);

print "CONST MAXPRIME = $maxprime;\n";
print "TYPE  PrimeMap = ARRAY [0..MAXPRIME] OF BITS 1 FOR [0..1];\n";
print "CONST Primes   = PrimeMap{0\n";

$width=0; $currprime=shift(@primes);
for ($i=1; $i<$maxprime; $i++) {
    #---check linewidth---
    if ($width>=$linewidth/2) {
      $width=0;
      print "\n";
    };
    $width++;

    #---print map---
    if ($i<$currprime) {
      print ",0";
    }
    else {
      print ",1";
    	$currprime=shift(@primes);
    };
}; 

print "\n};\n";
select(STDOUT);

#---fini
close(OUTFILE);
print "min=$minprime, max = $maxprime\n";
\end{verbatim} \end{tt}

Numbers beyond 46933 are checked by trying to factor them.
If any factor is found, we return FALSE immediately.  If nothing is
found up to the number's sqrt, then we return TRUE.

\subsection*{Factoring: factor}
[HGG: This is a crude algorithm.  Surely there are better approaches
available.]  We start with the lowest prime and determine (via mod)
whether or not it divides evenly into the number being examined.
If it is divisible, we collect as many repititions of that factor
as we can, and then go on to the next prime.  As we go, the number being
examined is being divided by its factors, so it is getting smaller.
We continually check to see if it has become smaller than the next prime to
be checked.  When that is true, we quit.
Due to the halting criterion, primes are reported as themselves
to the 1st power.

We could probably add a test for exceeding the
maximum prime in the prime map.  In that case, we are (I think)
dealing with a prime in the range $\sqrt{2^{31}} \ldots 2^{31}$.    

\subsection*{Factoring: gcd}
Greatest Common Demoninator is determined by Euclid's algorithm, which
is described in many sources.  We will reference Sedg88, pp 8-9.
Sedgewick gives an implementation in Pascal, and mentions an improvement
using mod.  We have implemented the improvement.

The idea is to find the largest factor (prime or not) which divides evenly
both the numerator and denominator of a fraction.  E.g.:
  $$\frac{12}{18}=\frac{2*6}{3*6}\mbox{\  thus\  }gcd=6$$

\subsection*{sqrt}
This comes from Hein96.  His example 2b was translated to
Modula-3.  Heinrich also provides several assembler implementations.

The idea is find $\sqrt{N}$ as the sum of two numbers: $N=
(u+v)^2=u^2+2uv+v^2$.  P. Heinrich notes that chosing $u$ and $v$
to fit base 2 operations (e.g., shift) allows a fast implementation.
Read his analysis for the full discussion.

The approach is based on recognition that
$\lfloor \log_2{N}\rfloor$ is found from the highest set bit.  And
$\lfloor \log_2{\sqrt{N}}\rfloor = \lfloor\log_2N^{1/2}\rfloor
= \lfloor 1/2\log_2{N}\rfloor$
is the highest set bit in $\lfloor \sqrt{N}\rfloor$.  It also depends on
the fact that if a number {\tt v} is a power of 2 such that it
has only one bit, {\tt vbit}, set, then we can multiply any number
by {\tt v} just be leftshifting {\tt vbit} times. 

Given this, let $u=2^{\lfloor 1/2\log_2{N}\rfloor}$, so that u is the value
of the highest set bit.  Let $v$ be the value of the next lower bit.
Then calc $u^2+2uv+v^2$.  If this comes out $\le N$ then (at least) the
next lower bit must be set.  In that case let $u:=(u+v)$ and
go again.  The payoff is in being able to compute $u^2+2uv+v^2$ using
addition and shifts.  If we take $2uv+v^2 = v(2u+v)=v(u+u+v)$, we
need leftshift(u+u+v,vbit).  NOTE: Heinrich uses {\tt l2} for vbit,
but that is easily confused with the number twelve, and the significance
may be misunderstood.  In our implementation, we use vbit and we 
reorder calculations to emphasize the importance of vbit and v.

How do we find vbit in the first place?  
We could start at 31 and leftshift 
until $u<0$, indicating that the 31st bit was set.  Or we could
start at 0 and rightshift until $u=0$, indicating there were no
more set bits. Which one works best depends on where vbit
resides on average.  We often use integer sqrt's in two places:
\begin{enumerate}
\item Graphics, where one might compute a distance as $\sqrt{x^2+y^2}$.  
Given x and y in the range 0..1000, the sum is 0..1,000,000, or
vbit up to 20.
\item Data analysis, where one might compute $\sqrt{1/N\sum{(\bar{x}-x_i^2)}}$.  
Data values here might be from 16-bit ADC's, with an average at mid range
or vbit around 15.  
\end{enumerate}

This suggests that working from 0 upward is a good approach.  
See the code comments for details.

Now that we have an approach, what is the interface?  Heinrich doesn't
statistice a specific domain in his article, but experimentation shows
that the algorithm cannot handle inputs whose sqrts are 
larger than half the bits in the word.  In other words, for a 32-bit word, 
$\sqrt{N} < 2^{16}$.  Examining the algorithm, we see this is because 
u2 is obtained by leftshifting by vbits twice.  If vbit is over
halfway, we end up shifting data out of the word.
Thus, we need to place a range limitation on the inputs.

\subsection*{sin\_cos}
Coordinate Rotation Digital Computer (CORDIC) began with Volder in 1959.
We followed Jarv90, Bert92, and Turkowski's article in Gems90 (pg
494 and code on pg 773). While CORDICs can be used for other 
transcendentals (see Jarv90), we just do sin and cos.  Mainly we
follow Bert92, though with bits=16 instead of 14.

The idea is to approximatrixe transcendental functions in ways that
require only integer-type operations such as shift and add.  The analysis
leads to the following equations:
\begin{eqnarray}
\mbox{bits} & = & \mbox{(whatever you choose)}\\
\mbox{base} & = & 2^{\mbox{bits}}\\
\mbox{radians\_to\_cordic} & = & \frac{\mbox{base}}{\pi / 2}\\
\mbox{cordic\_to\_real} & = & \frac{1}{\mbox{base}}\\
\mbox{expansion\_factor} & = & \prod_{i=0}^{\mbox{bits}}
      \sqrt{\frac{2^{2 \cdot i}+1}{2^{2 \cdot i}}}\\
\mbox{init\_val} & = & \frac{1}{\mbox{expansion\_factor}}\\ 
j & = & 0 \ldots \mbox{bits}\\
\mbox{atan\_table} & = & \arctan (\frac{1}{2^j})
      \cdot \mbox{radians\_to\_cordic}
\end{eqnarray}

It all starts with the number of bits.  We chose 16 as a convenient
number.  That gives 64K increments per quadrant or 256K around the 
circle.  The numbers were computed in MathCad and copy-and-pasted
into xInteger.m3. We used Turkowski's idea of hardcoding the
arctan table.  We used Bertrandom's idea of presetting x to a correction
factor prior to starting work.

The algorithm works by trying to coax the angle theta to 0, by
adding and subtracting smaller and smaller angles, obtained from the
arctan table. In lock step, we adjust x and y by the associated amounts
via shifting.  In the end, when theta reaches 0, x and y will be the 
cos and sin in CORDIC units.  If we wish, we can divide the x and y values
by the base to get real numbers values, though this step is not
needed in an all-integer project. 

The basic algorithm works in the first quadrant.  Jarvis warns that 
the system doesn't always converge in other quadrants.  Bertrandom 
works only in the first quadrant, and adjusts other values into
this quadrant.  We followed Bertrandom, but had trouble with his actual
quadrant decomposition.  As coded in xInteger.m3, it works.  [HGG:
I wonder if there was a typo in the published article.]
